\documentclass{article}
\usepackage[utf8]{inputenc}
\usepackage{chemfig}
\usepackage{tikz}
\usetikzlibrary{arrows}
\usepackage{relsize}
\usepackage[paperwidth=9.5in,paperheight=4in]{geometry}


\begin{document}

\startscheme
    \chemfig{=^[:120.0,1.0](-[:60.0,1.0]{\lewis{7:,N}}(-[:120.0,1.0]@{oxy}{\lewis{0:2:4:,O}^{^-}})-[:0.0,1.0]{\lewis{2:6:,O}H})-[:180.0,1.0]=^[:-120.0,1.0]-[:-60.0,1.0](-[:-120.0,1.0])=^[:-0.0,1.0]-[:60.0,1.0]-[:0.0,1.0]C_7H_6NO_2}
    + 
    \chemfig{@{hcah}H-[@{hcab}:0.0,1.0]@{hcacl}Cl}
    +
    \chemfig{\lewis{,Sn}}^{^{2+}}
    +\hphantom{\mid}
    \chemfig{\lewis{0:2:4:6:, Cl}^{^-}}
    \mathlarger{\mathlarger{\mathlarger{\mathlarger{\longrightarrow}}}}
    \chemfig{=^[:120.0,1.0](-[:60.0,1.0]{\lewis{7:,N}}(-[:120.0,1.0]{H\lewis{0:2:,O}})-[:0.0,1.0]{\lewis{2:6:,O}H})-[:180.0,1.0]=^[:-120.0,1.0]-[:-60.0,1.0](-[:-120.0,1.0])=^[:-0.0,1.0]-[:60.0,1.0]-[:0.0,1.0]C_7H_6NO_2}
    + \chemfig{\lewis{,Sn}}^{^{2+}}
    + \hphantom{|}
    2\hphantom{2}\chemfig{\lewis{0:2:4:6:, Cl}^{^-}}
    \hphantom{x}
\stopscheme
\chemmove[->]{
    \draw(oxy)..controls+(0:30mm)and+(90:15mm)..(hcah);
    \draw(hcab)..controls+(270:5mm)and+(195:5mm)..(hcacl);
}

%I have literally taken 5 hours to do approximately 2% of the total mechanism. I am not going to continue this. I'll TeX the large reactions, and I'm going to do the rest by hand. 


\end{document}