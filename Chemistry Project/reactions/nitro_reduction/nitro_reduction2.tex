\documentclass{article}
\usepackage[utf8]{inputenc}
\usepackage{chemfig}
\usepackage{tikz}
\usetikzlibrary{arrows}
\usepackage{relsize}
\usepackage[paperwidth=9in,paperheight=4in]{geometry}


\begin{document}

\startscheme
    \chemfig{=^[:120.0,1.0](-[:60.0,1.0]@{n}N^{^\oplus}(=[@{nodb}:120.0,1.0]@{oxy}{\lewis{2:4:,O}})-[:0.0,1.0]{\lewis{2:6:,O}H})-[:180.0,1.0]=^[:-120.0,1.0]-[:-60.0,1.0](-[:-120.0,1.0])=^[:-0.0,1.0]-[:60.0,1.0]-[:0.0,1.0]C_7H_6NO_2}
    +
    \hphantom{\mid}
    \chemfig{@{tin}\lewis{4:,Sn}}
    +
    \chemfig{\lewis{0:2:4:6:, Cl}^{^-}}
    \mathlarger{\mathlarger{\mathlarger{\mathlarger{\longrightarrow}}}}
    \chemfig{=^[:120.0,1.0](-[:60.0,1.0]N(-[:120.0,1.0]{\lewis{0:2.4:,O}})-[:0.0,1.0]{\lewis{2:6:,O}H})-[:180.0,1.0]=^[:-120.0,1.0]-[:-60.0,1.0](-[:-120.0,1.0])=^[:-0.0,1.0]-[:60.0,1.0]-[:0.0,1.0]C_7H_6NO_2}
    + \hphantom{|}
    \chemfig{\lewis{4.,Sn}^{+}}
    + \hphantom{|}
    \chemfig{\lewis{0:2:4:6:, Cl}^{^-}}
    \hphantom{x}
\stopscheme
\chemmove[-left to]{
    \draw(tin)..controls+(170:10mm)and+(315:5mm)..(n);
    \draw(nodb)..controls+(30:5mm)and+(0:3mm)..(oxy);
    \draw(nodb)..controls+(30:5mm)and+(60:3mm)..(n);
}



\end{document}